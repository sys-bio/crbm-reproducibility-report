\documentclass{article}
\usepackage[margin=2cm,headheight=50pt,includeheadfoot]{geometry}
\usepackage{graphicx}
\usepackage{helvet}
\usepackage[labelfont=bf]{caption}
\usepackage{float}
\usepackage{inconsolata}
\usepackage[backend=bibtex,style=ieee]{biblatex}
\usepackage{amssymb}
\usepackage[dvipsnames]{xcolor}
\usepackage[breakable]{tcolorbox}
\usepackage{mathtools}
\usepackage{hyperref}
\usepackage{titling}
\usepackage[none]{hyphenat} % turn this on if you want to remove hyphenation
\usepackage{bbding}
\usepackage{pifont}
\usepackage{wasysym}
\usepackage{amssymb}
\usepackage{enumitem,amssymb}
\usepackage{pifont}
\usepackage{tikz}
\usetikzlibrary{shapes}

% Define new commands for checkboxes/ checkbox colors
\newlist{todolist}{itemize}{2}
\setlist[todolist]{label=$\square$}
\newcommand{\absent}{\raisebox{0pt}{\tikz{\node[draw,scale=0.7,regular polygon, regular polygon sides=4,fill=none](){};}}}
\newcommand{\present}{\raisebox{0pt}{\tikz{\node[draw,scale=0.7,regular polygon, regular polygon sides=4,fill=black!20!Cerulean](){};}}}

% Define link colors
\renewcommand{\familydefault}{\sfdefault}
\date{}
\usepackage{fancyhdr}
\setcounter{secnumdepth}{0} % sections are level 1
\hypersetup{
    colorlinks=true, % make the links colored
    linkcolor=blue, % color TOC links in blue
    citecolor=blue,
    urlcolor=blue, % color URLs in blue
    linktoc=all % 'all' will create links for everything in the TOC
    }

% Define fancy header 
\pagestyle{fancy}
\fancyhf{}
\fancyhead[R]{\textbf{Director:} Professor Herbert M. Sauro \\ University of Washington, Seattle, WA \\ \href{https://reproduciblebiomodels.org}{https://reproduciblebiomodels.org}}
\fancyhead[L]{\includegraphics[width=5cm]{\VAR{logo_image}}}
\rfoot{\thepage}


\begin{document}

\noindent
\textbf{Reproducibility report for:} \VAR{title}
\\
\textbf{Submitted to:} \VAR{journal}
\\
\textbf{Manuscript number/identifier:} \VAR{journal_submission_id}

\bigskip
\noindent
\textbf{Curation outcome summary:} \VAR{curation_outcome_summary}
\vspace{5mm}
\begin{tcolorbox}[breakable,height fill,
colback=white,
arc=0pt,
outer arc=0pt,
colframe=white,
top=2mm,
toptitle=2mm,
bottomtitle=2mm,
colbacktitle=white!80!black,
colframe=black,
coltitle=black, 
title= \textbf{Box 1:} Criteria for repeatability and reproducibility]
$\VAR{rubric[0][0]}$ \textbf{Model source code provided:}
\begin{todolist}
  \item [\present] Source code: a standard procedural language is used (e.g. MATLAB, Python, C)
  \begin{todolist}
      \item[\present] There are details/ documentation on how the source code was compiled
      \item[\absent] There are details on how to run the code in the provided documentation
      \item[\absent] The initial conditions are provided for each of the simulations
      \item[\absent] Details for creating reported graphical results from the simulation results
  \end{todolist}
  
  \item[\present] Source code: a declarative language is used (e.g. SBML, CellML, NeuroML)
  \begin{todolist}
    \item[\absent] The algorithms used are defined or cited in previous articles
    \item[\absent] The algorithm parameters are defined
    \item[\absent] Post-processing of the results are described in sufficient detail
  \end{todolist}
\end{todolist}
$\VAR{rubric[0][1]}$ \textbf{Model source code not provided:}
\begin{todolist}
    \item[\absent] The model is executable without source (e.g. desktop application, compiled code, online service)
    \begin{todolist}
        \item[\absent] There are sufficient details to repeat the required simulation experiments
    \end{todolist}
\end{todolist}
$\VAR{rubric[0][2]}$ \textbf{The model is described mathematically in the article(s):}
\begin{todolist}
    \item[\present] Equations representing the biological system
    \item[\absent] There are tables of parameter values
    \item[\absent] There are tables of initial conditions
    \item Machine-readable tables of parameter values
    \item Machine-readable tables of initial conditions
\end{todolist}
$\VAR{rubric[0][3]}$ \textbf{The simulation experiments using the model are described mathematically in the article:}
\begin{todolist}
    \item[\absent] Integration algorithms used are defined
    \item[\absent] Stochastic algorithms used are defined
    \item[\absent] Random number generator algorithms used are defined
    \item Parameter fitting algorithms are defined
    \item The paper indicates how the algorithms yield the desired output
\end{todolist}
\end{tcolorbox}


\begin{tcolorbox}[
colback=white,
arc=0pt,
outer arc=0pt,
colframe=white,
top=2mm,
toptitle=2mm,
bottomtitle=2mm,
colbacktitle=white!80!black,
colframe=black,
coltitle=black, 
title= \textbf{Box 2:} Criteria for accessibility]

\begin{todolist}
  \item[\VAR{rubric[1][0]}] Model/source code is available at a public repository or researcher's web site
  \begin{todolist}
    \item[\absent] Prohibitive license provided
    \item[\absent] Permissive license provided
    \item[\absent] Open-source license provided
  \end{todolist}
  \item[\VAR{rubric[1][1]}] All initial conditions and parameters are provided
  \item[\VAR{rubric[1][2]}] All simulation experiments are fully defined (events listed, collection times and measurements specified, algorithms provided, simulator specified, etc.)
\end{todolist}
\end{tcolorbox}


\begin{tcolorbox}[
colback=white,
arc=0pt,
outer arc=0pt,
colframe=white,
top=2mm,
toptitle=2mm,
bottomtitle=2mm,
colbacktitle=white!80!black,
colframe=black,
coltitle=black, 
title= \textbf{Box 3:} Rules for Credible practice of
 Modeling and Simulation\footnote{Model credibility is assessed using the Interagency Modeling and Ananlysis Group conformance rubric:\\ \href{https://www.imagwiki.nibib.nih.gov/content/10-simple-rules-conformance-rubric}{https://www.imagwiki.nibib.nih.gov/content/10-simple-rules-conformance-rubric}}]
\begin{todolist}
    \item[
        %% if rubric[2][0] == 'Insufficient'
            \absent
        %% else
            \present
        %% endif
    ] Define context clearly: \VAR{rubric[2][0]}
    \item[
        %% if rubric[2][1] == 'Insufficient'
            \absent
        %% else
            \present
        %% endif
    ] Use appropriate data: \VAR{rubric[2][1]}
    \item[
        %% if rubric[2][2] == 'Insufficient'
            \absent
        %% else
            \present
        %% endif
    ] Evaluate within context: \VAR{rubric[2][2]}
    \item[
        %% if rubric[2][3] == 'Insufficient'
            \absent
        %% else
            \present
        %% endif
    ] List limitations explicitly: \VAR{rubric[2][3]}
    \item[
        %% if rubric[2][4] == 'Insufficient'
            \absent
        %% else
            \present
        %% endif
    ] Use version control: \VAR{rubric[2][4]}
    \item[
        %% if rubric[2][5] == 'Insufficient'
            \absent
        %% else
            \present
        %% endif
    ] Document adequately: \VAR{rubric[2][5]}
    \item[
        %% if rubric[2][6] == 'Insufficient'
            \absent
        %% else
            \present
        %% endif
    ] Conform to standards: \VAR{rubric[2][6]}
\end{todolist}
\end{tcolorbox}

\begin{tcolorbox}[
colback=white,
arc=0pt,
outer arc=0pt,
colframe=white,
top=2mm,
toptitle=2mm,
bottomtitle=2mm,
colbacktitle=white!80!black,
colframe=black,
coltitle=black, 
title= \textbf{Box 4:} Evaluation]
\begin{todolist}
    \item[\VAR{rubric[3][0]}] Model and its simulations could be repeated using provided declarative or procedural code
    \item[\VAR{rubric[3][1]}] Model and its simulations could be reproduced
\end{todolist}
\end{tcolorbox}


\newpage
\bigskip
\noindent
\textbf{Summary comments:} \VAR{summary_comments}

% \bigskip
% \noindent
% Could be several paragraphs needed to describe this so 

\bigskip
\bigskip
\bigskip
\begin{figure}[ht]
\begin{minipage}[b]{0.45\linewidth}
\centering
Anand K. Rampadarath\footnotemark, PhD
\\
Curator
\\
Center for Reproducible Biomedical Modeling
\end{minipage}
\hspace{0.5cm}
\begin{minipage}[b]{0.45\linewidth}
\centering
David P. Nickerson, PhD
\\
Curation Service Director
\\
Center for Reproducible Biomedical Modeling
\end{minipage}
\end{figure}
\footnotetext{Email: a.rampadarath@auckland.ac.nz}
\begin{figure}[ht]
\centering
Auckland Bioengineering Institute,
\\
University of Auckland
\end{figure}






\end{document}